\documentclass[landscape]{a0poster} 

% encoding and font
\usepackage[utf8]{inputenc}
\usepackage[T1]{fontenc}
\usepackage{lmodern}
\usepackage[english]{babel}

% Geometry
\usepackage{geometry}
\geometry{top=2cm, bottom=2cm, left=2cm, right=2cm}

% Fonts and visuals
\usepackage{graphicx, color, xcolor}
\usepackage{multicol}
\usepackage{amsmath, amssymb}
\usepackage{booktabs}
\usepackage{enumitem}
\usepackage{titlesec}
\usepackage{caption}
\usepackage{pgfplots}
\pgfplotsset{compat=1.18}

% Charts and diagrams
\usepackage{tikz}
\usepackage{pgf-pie}

% Fonts
\usepackage{sourcecodepro}
\renewcommand{\familydefault}{\sfdefault}

% Header
\usepackage{fancyhdr}
\pagestyle{fancy}
\fancyhf{}
\lhead{BitTorrent Distributed Systems}
\rhead{Leon Example}
\renewcommand{\headrulewidth}{1pt}

% Section formatting
\titleformat{\section}{\Large\bfseries\color{blue}}{}{0em}{}
\titleformat{\subsection}{\large\bfseries\color{black}}{}{0em}{}

% URL support
\usepackage{hyperref}

% Load custom layout helpers
\usepackage{PosterTemplates}

\begin{document}

% Title Banner
\maketitlebar{How BitTorrent Optimises Distributed Systems for Peer-to-Peer File Transfers}

\begin{multicols*}{3}

  \addsection{What is BitTorrent?}
BitTorrent is a peer-to-peer protocol and software for sending files over the internet, which allows users to send large files over the internet by connecting multiple sources at the same time, rather than relying on a single server to do it instead.

\addsection{Why do we need it? Limitations of traditional file transfer}
Traditional methods like email or FTP lack any security and have limited data sharing. This can lead to data inconsistency and redundancy. These methods also struggle with larger file sizes and complex data sharing. Other methods like using centralised servers, also introduce problems like single points of failure, potential data loss or corruption and many more.

\addsection{Importance of distributed systems and P2P architecture}
Distributed systems are an essential part for today’s modern computing needs, with their abilities such as enhanced scalability, fault tolerance and other factors that make them great for handling large-scale data transfer and processing. The P2P architecture allows our distributed systems to eliminate central authority by allowing nodes to function both as clients and servers, which then improves the systems resilience to faults as there no longer is a single point of failure. It also further improves scalability as new nodes are able to seamlessly be joined into the network.

\addfigure{img/placeholder.png}{Overview of P2P systems.}

\addsection{How BitTorrent Works}
Bittorrent works by locating other computers using the same software that have the file you wish to download. These other computers are ordinary computers like your own that are known as peers.
The process starts by finding a link for the file you want. Your bittorrent client software will then communicate with a tracker to find other computers running BitTorrent that have the complete file (known as seed computers) and those with portions of the file, these are typically peers in the process of downloading said file. The tracker identifies the swarm, which is a group of connected computers that have all or a portion of the file and are also in the process of sending or receiving it. The tracker then helps trade pieces of the file you want with others in the swarm, you can also receive multiple pieces of the files at once.

\addfigure{img/placeholder.png}{Swarm communication between peers.}

\addsection{Optimisations in BitTorrent}
\begin{itemize}[leftmargin=*]
  \item \textbf{Parallel Downloads:} Multiple pieces from multiple peers.
  \item \textbf{Rarest First:} Ensures availability of all pieces.
  \item \textbf{Tit-for-Tat:} Rewards uploaders.
  \item \textbf{Choking/Unchoking:} Dynamically prioritizes peers.
\end{itemize}
\addfigure{img/placeholder.png}{Peer lifecycle from leecher to seeder.}

\addsection{Performance Comparison}
\subsection*{Scalability}
Performance improves with more peers, unlike centralized systems.
\subsection*{Resilience}
Files are available even if some peers go offline.
\begin{barchart}{Scalability, Resilience, Efficiency}
\addplot coordinates {(Scalability,90) (Resilience,80) (Efficiency,95)};
\end{barchart}

\addsection{Real-World Applications}
\begin{itemize}[leftmargin=*]
  \item \textbf{Linux Distros:} Ubuntu, Fedora, Arch
  \item \textbf{Game Updates:} Used by Blizzard and Steam
  \item \textbf{Enterprise:} Facebook used it internally
\end{itemize}
\addpiechart{30/Seeders, 50/Leechers, 20/Trackers}{Peer Distribution in a Swarm}

\addsection{Challenges and Limitations}
\begin{itemize}[leftmargin=*]
  \item \textbf{NAT Traversal:} Connection issues across firewalls.
  \item \textbf{Initial Seeder Bottleneck:} Delays at startup.
  \item \textbf{Security:} File integrity requires verification.
  \item \textbf{Legal Issues:} Often associated with piracy.
\end{itemize}
\addinfobox{yellow!20}{Fun Fact: BitTorrent accounts for over 20\% of global internet traffic during peak hours.}

\addsection{Conclusion}
BitTorrent shows how P2P systems can outperform traditional models in scalability and efficiency. Future Web3 systems will likely build on similar architecture.

% references 
\nocite{*}
\footnotesize
\bibliographystyle{plain}
\bibliography{../Bibliography/References}


\end{multicols*}

\end{document}
